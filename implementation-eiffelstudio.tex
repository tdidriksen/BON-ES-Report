\section{EiffelStudio}
\label{implementation_eiffelstudio}
In continuation of the \textsc{ui} discussion in the previous section, this section will examine how the actual code behind this \textsc{ui} works. Also, how EiffelStudio acts to switch to the appropriate view, and how text inside the textual views are generated, will be analyzed. All classes in figure \ref{fig:extractor_structure} will be described on a conceptual level to give the reader an idea how views are handled. To get a deeper understanding it is suggest to study the source code itself.

\begin{figure}[H]
\centering
\includegraphics[scale=0.8]{images/es0.png}
\caption{View bar in EiffelStudio}
\label{fig:EiffelStudio0}
\end{figure}

In figure \ref{fig:EiffelStudio0} the bar for changing between the standard views of EiffelStudio is shown, from left to right, \textit{Basic text view}, \textit{Clickable view}, \textit{Flat view}, \textit{Contract view}, and \textit{Interface view}. Each of these views are represented by a subclass of \textsc{eb$\_$class$\_$text$\_$formatter}, and as such, to keep in line with EiffelStudios structure, any new views should also be. This formatter is responsible for displaying text in a text area. This concept is illustrated by figure \ref{fig:extractor_structure}, where it can be seen that a subclass of the class text formatter has been made for formal textual \bon. Obviously a similar class has been implemented for informal \bon. The change between these views is handled by a development window which purpose is to be a container for project tools, in this case a textual view. The \textsc{textual$\_$bon$\_$format$\_$tables} is a shared format table that contains meta information about the view selected by the user. The \textsc{text$\_$formatter$\_$decorator} is responsible for decorating text to be shown, and works as a mediator the output strategy (explained next) and the formatter. This text formatter decorator also works as the link from the internal representation back to EiffelStudio (described by figure \ref{fig:bon_extraction_4}), as it processes text given from the internal \bon{} representation as tokens. The output strategy's (in figure \ref{fig:extractor_structure}: \textsc{textual$\_$bon$\_$formal$\_$output$\_$strategy}) base function is to generate some sort of textual output based on an abstract syntax. In the case of the views already in place in EiffelStudio the output strategy works as a visitor that traverses through the abstract Eiffel syntax and generates decorated text that way. As it was decided not to use a visitor pattern for the textual \bon{} extractor, the output strategy's role was to initialize the meta-object (mentioned in section \ref{design-bon-extraction}) and then start the generation of the textual \bon.

\begin{figure}[H]
\centerline{
\includegraphics[scale=0.7]{images/BON-extractor-structure-large.png}
}
\caption{Structure of the BON extractor's interaction with EiffelStudio for formal BON.}
\label{fig:extractor_structure}
\end{figure}

\subsection{BON Syntax Highlighting}
At the heart of the syntax highlighter for textual \bon{} in EiffelStudio is the scanner class \textsc{editor\_textual\_bon\_scanner}. Like the textual \bon{} scanner used for type checking, this scanner class is also generated by the \textit{gelex} tool. The syntax highlighting scanner for textual \bon{} is a modified version of an already existing scanner for syntax highlighting made for the Eiffel editors in EiffelStudio. 

Whenever an element that needs highlighting is encountered, such as a keyword or a symbol, an appropriate subtype of the class \textsc{editor\_token} is instantiated by the scanner. These are the same token classes that EiffelStudio makes use of when highlighting Eiffel syntax. The highlighted text in a view is thus represented as a stream of tokens, each of them appropriately decorated with a color chosen through the preferences of EiffelStudio. For a uniform experience of switching between an Eiffel view and a textual \bon{} view, the colors for each of the different syntactic elements are the same for \bon{} as they are for Eiffel.

