\section{Informal BON}
One could wonder why type checking of informal textual \textsc{bon} is interesting, since many parts of it does not involve types, such as queries and commands, and all logical expressions are expressed through the semantics of strings. There are, however, some interesting elements to inspect to ensure the integrity of the system.

\paragraph{}
Creation charts are checked by making sure that all creators and created classes exist in the context. Furthermore it will also warn the user of possible duplicate entries in the creation chart. This was decided to not be an error as it is not outright faulty type usage, but rather unnecessary. As such, the user could have a reason for keeping these duplicate entries for semantic reasons, and thusly raising a type error, because of this seems pedantic. Similar, scenario charts, and event charts are also checked for duplicate entries by name. Again this is expressed a warning instead of an error.

\subsection{Structure}
\label{implementation-informal-structure}
Most importantly, any cluster must be in exactly one system chart or in exactly one cluster chart. The type checker ensures this by checking that the cluster is not in more than one chart, and that it is mentioned in at least one chart. If \textit{cluster occurrences} is defined as the number of times a cluster is mentioned in another cluster chart or in a system chart this can be expressed as such: 
\begin{figure}[H]
{\footnotesize\begin{center} $\neg$\textit{(cluster occurrences $\textgreater$ 1)} $\wedge$ \textit{cluster occurrences $\neq$ 0} \textit{} $\equiv$  \textit{cluster occurrences = 1}
\end{center}}
\end{figure}
The first step({\footnotesize$\neg$\textit{(cluster occurrences $\textgreater$ 1)}}) is done by the cluster having a reference to its parent (cluster) chart, and a flag indicating if a cluster chart is in a system chart called \textit{is\_in\_system\_chart}. When a cluster chart then tries to assign this reference, if it is already assigned to another chart, or the \textit{is\_in\_system\_chart} is set to \textit{true}, the cluster must be mentioned in more than one system or cluster chart, and thus there is a type error. This also ensures that a cluster cannot be mentioned twice in the same chart. Thereafter, it is checked if all clusters are in a chart ({\footnotesize\textit{cluster occurrences $\neq$ 0}}). Checking all clusters in the systems \textit{is\_in\_system\_chart} flags and references to their enclosing cluster charts does this. If any clusters reference is void, and the \textit{is\_in\_system\_chart} flag is set to \textit{false}, then no system or cluster chart mentions it. If both these checks pass then the cluster is mentioned in exactly one cluster ({\footnotesize\textit{cluster occurrences = 1}}). The only thing left to check is that the cluster is not a subcluster of itself.

Similar to clusters, classes must be in exactly in one cluster. This is checked in the same way as the as it is done for clusters, by letting a class have a reference to its enclosing cluster.