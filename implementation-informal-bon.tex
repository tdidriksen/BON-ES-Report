\subsection{Informal BON}
One could wonder why type checking of informal textual \textsc{bon} is interesting, since many parts of it does not have types, such as features, and all logical expressions are expressed through the semantics of strings. There are however some interesting elements to inspect to ensure the integrity of the system.

\paragraph{}
Creation charts are checked by making sure that all creators and created classes exist in the context. Furthermore it will also warn the user of possible duplicate entries in the creation chart. This was decided to not be an error as it is not outright faulty type usage, but rather unnecessary clutter. The user however could be having a reason to keep these duplicate entries for semantic reasons, and not passing the entire specification because of this seemed pedantic. Similar to creation charts, scenario charts are also checked for duplicate entries, but again it was decided to express it as a warning instead of an error. 

\subsubsection{Structure}
\label{implementation-informal-structure}
First off, any cluster must be in exactly one system chart or in exactly one cluster chart. The type checker does this by checking that the cluster is not in more than one chart, and that it is mentioned in at least one chart. If cluster occurrences is the number of times a cluster is mentioned in another cluster chart or in a system chart this can be expressed as such: 
{\footnotesize\begin{center} $\neg$\textit{(cluster occurrences $\textgreater$ 1)} $\wedge$ \textit{cluster occurrences $\neq$ 0} \textit{} $\equiv$  \textit{cluster occurrences = 1}
\end{center}}
\paragraph{}
The first step({\footnotesize$\neg$\textit{(cluster occurrences $\textgreater$ 1)}}) is done by the cluster having a reference to its parent (cluster) chart. When a system or cluster chart then tries to assign this reference, if it is already assigned to another chart the cluster must be mentioned in more than one system or cluster chart, and thus there is a type error. This also ensures that a cluster cannot be mentioned twice in the same chart. Thereafter, it is checked if all clusters are in charts ({\footnotesize\textit{cluster occurrences $\neq$ 0}}). Checking all clusters in the systems references to their system charts does this. If any clusters reference is void then no system or cluster chart mentions it. If both these checks pass then the cluster is mentioned in exactly one cluster ({\footnotesize\textit{cluster occurrences = 1}}). The only thing left to check is that a cluster does not contain itself, which is simply done by checking the clusters contained clusters for itself. 

Similar to clusters, classes must be in exactly in one cluster. This is checked in the same way as the as it was for clusters, through classes references to its parent cluster.