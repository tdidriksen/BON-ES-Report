\chapter{Conclusion} 
In this report it has been shown how textual \bon{} can be integrated into EiffelStudio. For displaying textual \bon{} in EiffelStudio, two additional views were implemented in the EiffelStudio GUI: an informal \bon{} view for displaying informal specifications and a formal \bon{} view for displaying formal specifications, respectively. These allow the user to easily obtain a \bon{} specification from his or her Eiffel source code at any given time during implementation. As such, the textual \bon{} tool supports the notion of reversibility, one of the main ideas of \bon{}, since the extracted specification is always up-to-date with the latest changes in the source code. The extractor was examined in chapter \ref{eiffelstudio-integration}

To achieve the goal of providing an overview of a system, the textual \bon{} extractor also analyzes and extract the descendants of the class in scope. This overview allows the user to inspect multiple classes within an inheritance hierarchy, without changing view. The extracted \bon{} is fairly complete, however with a few omissions (see appendix \ref{components-not-extracted}) and bugs (see section \ref{bugs_bon_extractors}). Overall it fulfills its purpose of giving an overview over a system.

A general-purpose type checker for textual \bon{} based on the grammar presented in Walden and Nerson's book was implemented. The type checker is based on a procedural pattern, in which the well-typedness of each of the elements/components of the grammar is evaluated by a designated feature. The type checking algorithm features a two-phased strategy, which ensures consistent type checking. Due to the underspecified nature of the notation, it was explained which rules were added to make it type check. A comprehensive list of these rules can be found in appendix \ref{appendix-type-system}. 

With these additions to the Eiffel and \bon{} worlds, a stepping stone for bridging the gap between these two has been laid. Other projects can, with basis in this one, help to further bridge this gap and let these two worlds work together.