\chapter{Introduction}
\label{sec:introduction}
When designing systems, most developers face a two-fold challenge of finding a common fashion of expressing the system and its components so that it is both understandable for people outside the software engineering world, and at the same time gives both a low- and high-level description of the system for the purpose of development. Many methods have been designed to solve this, one of which is the \bon{} method. \bon{} relies on both graphical and textual elements. When developing Eiffel in EiffelStudio, the user has a graphical \bon{} tool available, which he can quickly open for a graphical overview of inheritance and client relations of a given class. Furthermore, there is no tool for inspecting the textual aspect of \bon. If one wants a textual representation they must write it by hand themselves.

\section{Problem definition}
The purpose of this project must provide a way to statically extract textual \bon{} from Eiffel within the EiffelStudio environment. It should be easy to quickly switch from editing Eiffel source code to viewing textual \bon. Furthermore, a textual \bon{} type checker must be able to check the well-typedness of the extracted \bon{} and return an appropriate error message to the user when it is not. It should be implemented in such a way that it in the future can be inserted into EiffelStudio. \newline\newline In short the final product must be able to:
\begin{itemize}
 
  \item Provide a way to extract textual \bon{} in EiffelStudio
  \item Type check textual \bon{} and provide meaningful feedback
 
\end{itemize}

\section{The Product}
Writing specifications can be a strain, thus many of developers want to shortcut this work by generating the specification with tools. However, as mentioned above, there are no tools to extract \textsc{bon} from Eiffel, even thought \bon{} originates from the world of Eiffel. Doing so, is the exact purpose of this tool. A tool to extract Eiffel from \textsc{bon}, in the main Eiffel development environment, EiffelStudio. The idea is that a user at any given time can switch from writing Eiffel code, to reading a textual \bon{} specification for his work. Still, the extracted \bon{} should not just analyze one class, but also provide an overview of other relations of the class.

To ensure that the extracted textual \bon{} is syntactically correct and well-typed, a type checker is also implemented. This type checker will receive the \bon{} through an extended version of the \textsc{ebon} parser/lexer (\cite{ebon}). A this point this type checker is not yet integrated in EiffelStudio, but as it is fully implemented in Eiffel, it is the next logical step. 

\section{Related Work}

\subsection{Extended BON}
The Extended \bon{} Tool Suite (\textsc{ebon}) is a project whose objective is to add semantic properties into the \bon{} standard. Extended \bon{} is developed by Joseph Kiniry. The textual \bon{} parser and lexer used for the work described in this report is based on the parser and lexer from \textsc{ebon}, as well the abstract syntax as a meta object graph.

\subsection{BONc}
\bon{c} is a command-line textual \bon{} parser and type checker implemented in Java \cite{bonc}. The tool is developed and maintained by Fintan Fairmichael. \bon{c} provides support for generating graphical representations of a \bon{} specification, namely through generation of informal charts in HTML format and graphical \bon{} diagrams. The standard types defined in the type checker described in the report is based in the built-in types in \bon{c}.

\subsection{BON IDE}
The \bon{} Integrated Development Environment is a tool which puts emphasis on the graphical aspect of \bon{} \cite{bonide}. It provides tool for visualizing a \bon{} specification as a graphical diagram, and bases its work on the Ecore model of \bon{}.

\subsection{The BON CASE Tool}
The BON CASE Tool is a CASE (Computer-Aided Software Engineering) tool that supports the creation of \bon{} models and the formal reasoning of such \cite{boncase}. It supports integration of \bon{} with \textsc{jml} in order to be able to utilize the formal techniques and tools of \textsc{jml}. The BON CASE Tool is developed by Paige, Kaminskaya, Ostroff, and Lancaric.

\subsection{Other Papers}
\cite{ostroff2001} formulates a metamodel for \bon{} using PVS, which expresses the syntactic constraints that all models using \bon{} must obey. Some of the rules defined here have been implemented in the type checker described in this report.

\cite{ostroff1999} describes how the properties of \bon{} make the language a well suited tool for formal specifications and algorithm refinement.

\cite{ostroff1998} shows how a transition from the \textsc{z} formal notation to \bon{} can be made. Furthermore, Paige and Ostroff show how \bon{} has the expressive power of \textsc{z} with the added value of an object-oriented approach.
