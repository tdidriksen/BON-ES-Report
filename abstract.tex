\begin{abstract}
The Business Object Notation (\bon{}) is a method and notation for writing software specifications. The notation consists of two parts, a graphical and a textual. The purpose of this project is to bridge the gap between the textual \bon{} world and the world of Eiffel. There is already support for the graphical aspect of \bon{} in the main Eiffel \textsc{ide}, EiffelStudio, and as such involving the textual aspect seems like a natural step. Support for textual \bon{} in EiffelStudio should make it easier to get an overview over an Eiffel system, and should provide a higher level description than the code itself.

To achieve this, support for textual \bon{} has been implemented directly into EiffelStudio. It allows for seamless switching between Eiffel source code and extracted textual \bon. To  provide a way to verify the validity of the extracted \bon{} a general purpose type checker was developed. The idea is that this type checker at some point can be implemented into EiffelStudio and be an aid in \bon{} development in an Eiffel environment.

This has resulted in a tool in EiffelStudio that extract textual \bon{} from a class written in Eiffel. To provide an overview over a system, textual \bon{} for all descendants of the class is also extracted. In addition a general purpose type checker has been build upon an already existing parser and lexer. This type checker is not yet integrated in EiffelStudio.

Having this option in EiffelStudio it is now possible to get an overview of a system using \bon{}. However, this does not mean that this area is fully explored. There are many interesting features to be added, in example fully implement the type checker into EiffelStudio and extraction of a full system to textual \bon{} view in EiffelStudio. This project provides a starting point for hopefully many other projects that can continue the work on integrating textual \bon{} into EiffelStudio and fully bridge the gap between these two worlds.
\end{abstract}